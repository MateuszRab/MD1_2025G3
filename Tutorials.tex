\documentclass{mwart}

            % Papier
\usepackage{polski}
\usepackage[utf8]{inputenc}
\usepackage[a4paper, total={16cm, 24cm}]{geometry}

            % Matematyka
\usepackage{amsthm}
\usepackage{amsmath}
\usepackage{amssymb}

           % Wygląd
%\usepackage{url}
\usepackage{hyperref}
\usepackage{graphics}
\usepackage{fancyhdr}
\usepackage{geometry}
\usepackage{enumitem}



            % Ustawienia ramek
\usepackage[nobreak=false]{mdframed}
\mdfsetup{%
% frametitlerule=true,
% subtitlebelowline=true,
subtitleaboveline=true,
subtitleaboveskip=0,
subtitlebelowskip=0,
linewidth=1pt}


% Skroty
\newcommand{\R}{\mathbb{R}}
\newcommand{\N}{\mathbb{N}}
\newcommand{\Z}{\mathbb{Z}}
\newcommand{\C}{\mathbb{C}}

\newtheorem{zad}{Zadanie}[section]


% Początek strony tytułowej


\author{\textbf{Grupa 3} \\ \textsf{Informatyka i Systemy Informacyjne | MiNI PW}}
\title{%
%\includegraphics[width=1\textwidth]{logo_uczelni.png}
\textbf{Matematyka Dyskretna I}}


%\setcounter{secnumdepth}{1}
\pagestyle{fancy}
\setlength{\headsep}{1.2cm}
\fancyhead{}
\lhead{Grupa 3\\\textsf{Wydział MiNI PW}}
\chead{\Large \textbf{Rozwiązania zadań}}
\rhead{\textbf{Matematyka Dyskretna 1}\\Informatyka}

\fancyfoot{}
\rfoot{\texttt{\thepage}}

\fancypagestyle{specialfooter}{%
    \fancyhf{}
    \renewcommand\headrulewidth{0pt}
    \fancyfoot[C]{\Large{Lato 2025}}
}


\begin{document}
\maketitle{}
\begin{center}
    Studenckie rozwiązania zadań z ćwiczeń z Matematyki Dyskretnej I. \\
    Ćwiczenia z \emph{MD I} w gr. 3 prowadzi dr inż. Tomasz Brengos \\
    i też oto pod jego opieką powstaje ten plik. \\
    Ostatnia aktualizacja: \today{} \\
    \tableofcontents
    \thispagestyle{specialfooter}

\end{center}

% Koniec strony tytułowej

















% Poczatek dokumentu
\newpage
\section{Zestaw}              % Zestaw 1
\begin{zad}[Autor 1, Autor 2]
    Na płaszczyźnie poprowadzono $n$ prostych, z których żadne dwie nie
    są równoległe i żadne trzy nie przechodzą przez ten sam punkt.
    Wyznacz liczbę:
    \begin{enumerate}
        \item obszarów, na które te proste dzielą płaszczyznę;
        \item obszarów ograniczonych, na które te proste dzielą płaszczyznę.
    \end{enumerate}
\end{zad}
\begin{mdframed}
    \begin{enumerate}
        \item Rozwiązanie Autora 1 podpunktu 1
        \item Rozwiązanie Autora 1 podpunktu 2
    \end{enumerate}
\end{mdframed}
\begin{mdframed}
    \begin{enumerate}
        \item Rozwiązanie Autora 2 podpunktu 1
        \item Rozwiązanie Autora 2 podpunktu 2
    \end{enumerate}
\end{mdframed}



\begin{zad}[Autor 1, Autor 2]
    Ciąg Fibonacciego $\{F_n\}_{n \in \mathbb{N}}$ zadany jest przez
    $F_0=0$, $F_1=1$ i $F_{n+2}=F_{n+1}+F_n$. Udowodnij, że: \\
    \begin{enumerate}
        \item $F_0 + ... + F_n = F_{n+2} - 1$;
        \item $5|F_{5n}$,
        \item $F_n < 2_n$.
    \end{enumerate}
\end{zad}
\begin{mdframed}
    \begin{enumerate}
        \item Rozwiązanie Autora 1 podpunktu 1
        \item Rozwiązanie Autora 1 podpunktu 2
        \item Rozwiązanie Autora 1 podpunktu 3
    \end{enumerate}
\end{mdframed}
\begin{mdframed}
    \begin{enumerate}
        \item Rozwiązanie Autora 2 podpunktu 1
        \item Rozwiązanie Autora 2 podpunktu 2
        \item Rozwiązanie Autora 2 podpunktu 3
    \end{enumerate}
\end{mdframed}




\begin{zad}[Autor 1, Autor 2]
    Turniej $n$-wierzchołkowy to dowolny graf skierowany $G = (V, E)$, gdzie $|V| = n$
    i w którym $(u, v) \in E$ lub $(v, u) \in E$ dla dowolnych $u, v \in V$.
    Pokaż, że w dowolnym niepustym turnieju istnieje wierzchołek z którego można “przejść”
    po krawędziach zgodnie z ich skierowaniem do dowolnego innego wierzchołka w co
    najwyżej dwóch krokach.
\end{zad}
\begin{mdframed}
    Rozwiązanie Autora 1.
\end{mdframed}
\begin{mdframed}
    Rozwiązanie Autora 2.
\end{mdframed}

\begin{zad}[Autor 1, Autor 2]
    Udowodnij, że każdy turniej ma ścieżkę Hamiltona.
\end{zad}
\begin{mdframed}
    Rozwiązanie Autora 1.
\end{mdframed}
\begin{mdframed}
    Rozwiązanie Autora 2.
\end{mdframed}




\begin{zad}[Autor 1, Autor 2]
    W każdym polu szachownicy rozmiaru $n x n $ znajduje się jedna osoba.
    Część osób zarażona jest wirusem grypy. Wirus grypy rozprzestrzenia się w dyskretnych
    odstępach czasowych w sposób następujący:
    \begin{itemize}
        \item osoby zarażone pozostają zarażone,
        \item osoba ulega zarażeniu jeżeli co najmniej dwie sąsiadujące z nią osoby są już zarażone
              (przez osobę sąsiednią rozumiemy osobę siedzącą z przodu, z tyłu, z lewej lub prawej
              strony).
              Wykaż, że jeżeli na początku zarażonych jest istotnie mniej niż n osób, to w każdej chwili
              przynajmniej jedna osoba pozostaje niezarażona.
    \end{itemize}
\end{zad}
\begin{mdframed}
    Rozwiązanie Autora 1.
\end{mdframed}
\begin{mdframed}
    Rozwiązanie Autora 2.
\end{mdframed}




\begin{zad}[Autor 1, Autor 2]
    Wykaż, że w grupie $n$ osób istnieją dwie, które mają taka samą liczbę znajomych.
\end{zad}
\begin{mdframed}
    Rozwiązanie Autora 1.
\end{mdframed}
\begin{mdframed}
    Rozwiązanie Autora 2.
\end{mdframed}




\begin{zad}[Autor 1, Autor 2]
    Przy okrągłym stole jest $n$ miejsc oznaczonych proporczykami różnych
    państw. Ambasadorowie tych państw usiedli przy tym stole tak, że żaden z nich nie siadł
    przy właściwym proporczyku. Wykaż, że można tak obrócić stołem, że co najmniej 2
    ambasadorów znajdzie się przed proporczykiem swojego państwa.
\end{zad}
\begin{mdframed}
    Rozwiązanie Autora 1.
\end{mdframed}
\begin{mdframed}
    Rozwiązanie Autora 2.
\end{mdframed}




\begin{zad}[Autor 1, Autor 2]
    Pokaż, że w dowolnym ciągu n liczb całkowitych istnieje (niepusty)
    podciąg kolejnych elementów taki, że suma wyrazów podciągu jest wielokrotnością n.
\end{zad}
\begin{mdframed}
    Rozwiązanie Autora 1.
\end{mdframed}
\begin{mdframed}
    Rozwiązanie Autora 2.
\end{mdframed}




\begin{zad}[Autor 1, Autor 2]
    Rozważ dowolną rodzinę podzbiorów zbioru $n$--elementowego zawierającą
    więcej niż połowę wszystkich podzbiorów. Wykaż, że w tej rodzinie muszą być dwa zbiory
    takie, że jeden zawiera się w drugim.
\end{zad}
\begin{mdframed}
    Rozwiązanie Autora 1.
\end{mdframed}
\begin{mdframed}
    Rozwiązanie Autora 2.
\end{mdframed}



\begin{zad}[Autor 1, Autor 2]
    Dla $n$--elementowego zbioru $X$ rozważ pewną rodzinę jego podzbiorów
    $\mathcal{F}$, gdzie $|F| > n/2$ dla każdego $F \in \mathcal{F}$. Wykaż, że istnieje
    $x \in X$ należący do co najmniej połowy zbiorów z $\mathcal{F}$.
\end{zad}
\begin{mdframed}
    Rozwiązanie Autora 1.
\end{mdframed}
\begin{mdframed}
    Rozwiązanie Autora 2.
\end{mdframed}


\begin{zad}[Autor 1, Autor 2]
    Dana jest kwadratowa szachownica $n \times n$. Dla jakich wartosci $n\geq 1$
    możemy pokryć tę szachownicę kostkami wielkości $2 \times 2$ oraz $3 \times 3$.
\end{zad}
\begin{mdframed}
    Rozwiązanie Autora 1.
\end{mdframed}
\begin{mdframed}
    Rozwiązanie Autora 2.
\end{mdframed}


\begin{zad}[Autor 1, Autor 2]
    Dana jest kwadratowa szachownica $2n \times 2n$ z wyciętym jednym polem.
    Wykaż, że dla wszystkich wartości $n \geq 1$ możemy pokryć tę szachownicę kostkami w
    kształcie litery L (czyli kwadrat $2 \times 2$ bez jednego pola).
\end{zad}
\begin{mdframed}
    Rozwiązanie Autora 1.
\end{mdframed}
\begin{mdframed}
    Rozwiązanie Autora 2.
\end{mdframed}






















\newpage
\section{Zestaw}          % ZESTAW 2

\begin{zad}[Filip Sajko, Autor 2]
    Na ile sposobów można ustawić $n$ wież na szachownicy
    $n \times n$ tak, by żadne dwie nie znajdowały się w
    polu wzajemnego rażenia.
\end{zad}
\begin{mdframed}
    Starczy zauważyć, że dla każdej wieży wybieramy rząd i kolumnę
    w której się znajduje -- i tym samym zmniejsza liczbę dostępnych
    o jeden. Tak więc odpowiedź wynosi: \[n \cdot n \cdot (n-1) \cdot
        (n-1) \cdot ... \cdot 2 \cdot 2 \cdot 1 \cdot 1 =  n! \cdot n!\]
\end{mdframed}
\begin{mdframed}
    Rozwiązanie Autora 2.
\end{mdframed}




\begin{zad}[Filip Sajko, Autor 2]
    Na ile sposobów można ustawić $k$ wież na szachownicy $n \times m$
    tak, by żadne dwie nie znajdowały się w polu wzajemnego rażenia.
\end{zad}
\begin{mdframed}
    Zadanie analogiczne od poprzedniego - z tym, że zmienił nam się
    rozmiar planszy, a ponadto nie wypełniamy jej całej. Zasada
    pozostaje jednak ta sama. Na start jednak warto założyć, że
    $k \leq max\{n, m\}$ (choć w sumie jeżeli tak nie jest, to
    odpowiedź to 0). Mając to już za sobą:
    \[n \cdot m \cdot (n-1) \cdot (m-1) \cdot ... \cdot (n - k +1) \cdot (m -k +1)\]
    (wykonujemy mnożenie $k + k$ elementów -- stąd to $-k + 1$).
\end{mdframed}
\begin{mdframed}
    Rozwiązanie Autora 2.
\end{mdframed}




\begin{zad}[Filip Sajko, Autor 2]
    Znaleźć definicje rekurencyjne następujących ciągów:
    \begin{enumerate}
        \item $a(n)$ -- liczba słów długości $n$ nad alfabetem
              $\{0, 1\}$, które nie zawierają dwóch jedynek koło siebie.
        \item  $b(n)$ -- liczba różnych pokryć prostokąta o wymiarze
              $2 \times n$ dominami wymiaru $2 \times 1$.
    \end{enumerate}
\end{zad}
\begin{mdframed}
    \begin{enumerate}

        \item Oczywiście $a(1)=2$, $a(2)=3$.  Rozważmy słowo $n$ elementowe.
              Zauważamy, że jeżeli ono kończy sie ono zerem to poprzedzające słowo $n-1$
              elementowe jest dowolne. Jeżeli natomiast kończy się jedynką,
              to poprzedzające słowo $n-2$ elementowe jest dowolne
              (tak jakby cofamy się krok dalej by mieć dowolność).
              Stąd: $a(n)=a(n-1)+a(n-2)$.
        \item Analogicznie do poprzedniego. Jak wiemy $a(1) = 1$, $a(2) = 2$. Zastanówmy się nad $a(n)$:
              Rozważamy ciąg o długości $n$. Jeżeli na końcu jest blok poziomy,
              to wiemy że powstał on z ciągu długości $a(n-2)$.
              Jeżeli jest pionowy, to wiemy, że musiał on powstać z ciągu długości $n-1$.
              A stąd $a(n) = a(n-1) + a(n-2)$.
    \end{enumerate}
\end{mdframed}
\begin{mdframed}
    Rozwiązanie Autora 2.
\end{mdframed}




\begin{zad}[Autor 1, Autor 2]
    Ile rozwiązań ma równanie $x_1 + x_2+x_3+x_4 = 7$:
    \begin{enumerate}
        \item gdzie $x_i$ są liczbami naturalnymi?
        \item gdzie $x_i$ są dodatnimi liczbami naturalnymi?
    \end{enumerate}
\end{zad}
\begin{mdframed}
    \begin{enumerate}
        \item Rozwiązanie Autora 1 podpunktu 1
        \item Rozwiązanie Autora 1 podpunktu 2
    \end{enumerate}
\end{mdframed}
\begin{mdframed}
    \begin{enumerate}
        \item Rozwiązanie Autora 2 podpunktu 1
        \item Rozwiązanie Autora 2 podpunktu 2
    \end{enumerate}
\end{mdframed}




\begin{zad}[Autor 1, Autor 2]
    Rozważmy czekoladę złożoną z $m\times n$ kostek.
    Na ile sposobów można wykroić prostokąt złożony z $k \times k$
    sąsiadujących ze sobą kostek czekolady?
\end{zad}
\begin{mdframed}
    Rozwiązanie Autora 1.
\end{mdframed}
\begin{mdframed}
    Rozwiązanie Autora 2.
\end{mdframed}




\begin{zad}[Autor 1, Autor 2]
    (Reguła sumowania po górnym indeksie). Udowodnij, że dla
    $n, k \in \mathbb{N}$ zachodzi
    \[\sum_{j=0}^n\binom{j}{k} = \binom{n+1}{k+1}\]
\end{zad}
\begin{mdframed}
    Rozwiązanie Autora 1.
\end{mdframed}
\begin{mdframed}
    Rozwiązanie Autora 2.
\end{mdframed}




\begin{zad}[Autor 1, Autor 2]
    (Reguła sumowania równoległego). Udowodnij, że dla $n, k \in \mathbb{N}$
    zachodzi \[ \sum_{j= 0}^{n}\binom{n+j}{j} = \binom{n+k+1}{k}   \]
\end{zad}\
\begin{mdframed}
    Rozwiązanie Autora 1.
\end{mdframed}
\begin{mdframed}
    Rozwiązanie Autora 2.
\end{mdframed}



\begin{zad}[Autor 1, Autor 2]
    Ile jest funkcji $f:\{1, ..., n\} \to \{1, ..., n\}$ monotonicznych takich,
    że $f(i) \leq f(j) $ dla $i < j$?
\end{zad}
\begin{mdframed}
    Rozwiązanie Autora 1.
\end{mdframed}
\begin{mdframed}
    Rozwiązanie Autora 2.
\end{mdframed}




\begin{zad}[Autor 1, Autor 2]
    Ile jest $k$--elementowych podzbiorów zbioru $n$--elementowego, które nie
    zawierają dwóch sąsiednich liczb?
\end{zad}
\begin{mdframed}
    Rozwiązanie Autora 1.
\end{mdframed}
\begin{mdframed}
    Rozwiązanie Autora 2.
\end{mdframed}


\begin{zad}[Autor 1, Autor 2]
    Posługując się interpretacją kombinatoryczną udowodnij, że:
    \[ \sum_{i=0}^{k} \binom{n}{i} \binom{n-i}{k-i} = 2^k \binom{n}{k} \]
\end{zad}
\begin{mdframed}
    Rozwiązanie Autora 1.
\end{mdframed}
\begin{mdframed}
    Rozwiązanie Autora 2.
\end{mdframed}




\begin{zad}[Autor 1, Autor 2]
    Udowodnij poniższe tożsamości na dwa sposoby: posługując się interpretacją
    kombinatoryczną albo rozwinięciem dwumianu $(1 + x)^n$:
    \begin{enumerate}
        \item \[\sum_{k=0}^{n}k\binom{n}{k} = n2^{n-1}\]
        \item \[\sum_{k=0}^{n}k^2\binom{n}{k}= (n+n^2)2^{n-2}\]
        \item \[\sum_{i=0}^{k}\binom{m}{i}\binom{n}{k-i} = \binom{m+n}{k} \]
    \end{enumerate}
\end{zad}
\begin{mdframed}
    \begin{enumerate}
        \item Rozwiązanie Autora 1 podpunktu 1
        \item Rozwiązanie Autora 1 podpunktu 2
        \item Rozwiązanie Autora 1 podpunktu 3
    \end{enumerate}
\end{mdframed}
\begin{mdframed}
    \begin{enumerate}
        \item Rozwiązanie Autora 2 podpunktu 1
        \item Rozwiązanie Autora 2 podpunktu 2
        \item Rozwiązanie Autora 2 podpunktu 3
    \end{enumerate}
\end{mdframed}






















\newpage
\section{Zestaw}          % ZESTAW 3

\begin{zad}[Autor 1, Autor 2]
    Wykaż, że dla dowolnego $n \geq 1$ istnieje $k \geq 1$ takie, że:
    \[S(n, 0) < S(n, 1) < ... < S(n, k - 1 ) \leq S(n, k) > S(n, k+1) > ... > S(n, n)\]
\end{zad}
\begin{mdframed}
    Rozwiązanie Autora 1.
\end{mdframed}
\begin{mdframed}
    Rozwiązanie Autora 2.
\end{mdframed}




\begin{zad}[Autor 1, Autor 2]
    Wykaż, że:
    \[B(n) = \sum_{i=0}^{n-1} \binom{n-1}{i}B(i)\]
\end{zad}
\begin{mdframed}
    Rozwiązanie Autora 1.
\end{mdframed}
\begin{mdframed}
    Rozwiązanie Autora 2.
\end{mdframed}




\begin{zad}[Autor 1, Autor 2]
    Wykaż, że dla $n, k \in \mathbb{N}$ zachodzi:
    \[S(n,k+1)=\frac{1}{(k+1)!} \sum_{0<i_0<...<i_{k-1}<n} \binom{n}{i_{k-1}}\binom{i_{k-1}}{i_{k_2}}...\binom{i_1}{i_0}     \]
\end{zad}
\begin{mdframed}
    Rozwiązanie Autora 1.
\end{mdframed}
\begin{mdframed}
    Rozwiązanie Autora 2.
\end{mdframed}





\begin{zad}[Autor 1, Autor 2]
    Rozważ  następującą procedurę generującą pewne liczby naturalne
    $\{a_{i,j}\}_{1 \geq i \geq j}$:
    \begin{enumerate}
        \item $a_{0,0} = 1$,
        \item $a_{n+1, 0} = a_{n,n}$, dla $n \geq 0$,
        \item $a_{n+1, k+1} = a_{n, k} + a_{n+1, k}$, dla $n \geq k \geq 0$.
    \end{enumerate}
\end{zad}
\begin{mdframed}
    Rozwiązanie Autora 1.
\end{mdframed}
\begin{mdframed}
    Rozwiązanie Autora 2.
\end{mdframed}




\begin{zad}[Autor 1, Autor 2]
    Wykaż, że liczba podziałów zbioru $(n - 1)$  elementowego jest równa
    liczbie podziałów zbioru $\{1, ..., n\}$ niezawierających sąsiednich liczb w jednym bloku.
\end{zad}
\begin{mdframed}
    Rozwiązanie Autora 1.
\end{mdframed}
\begin{mdframed}
    Rozwiązanie Autora 2.
\end{mdframed}




\begin{zad}[Autor 1, Autor 2]
    Udowodnij, że liczba ukorzenionych drzew binarnych na $n$ wierzchołkach to $n$-ta liczba Catalana.

    Ukorzenione drzewo jest drzewem binarnym, jeśli każdy wierzchołek ma co najwyżej
    dwójkę dzieci przy czym co najwyżej jedno lewe dziecko i co najwyżej jedno prawe dziecko.
\end{zad}
\begin{mdframed}
    Rozwiązanie Autora 1.
\end{mdframed}
\begin{mdframed}
    Rozwiązanie Autora 2.
\end{mdframed}




\begin{zad}[Autor 1, Autor 2]
    Triangulacją $n$ -- wierzchołkowego wielokąta wypukłego nazywamy zbiór
    $(n - 3)$ wzajemnie nieprzecinających się jego przekątnych, które dzielą jego obszar na
    $(n - 2)$ trójkątów.
    \begin{enumerate}
        \item ile jest triangulacji $n$--wierzchołkowego wielokąta wypukłego?
        \item Ile jest triangulacji $n$--wierzchołkowego wielokąta wypukłego, w których każdy trójkąt
              triangulacji ma przynajmniej jeden bok na brzegu wielokąta?
    \end{enumerate}
\end{zad}
\begin{mdframed}
    \begin{enumerate}
        \item Rozwiązanie Autora 1 podpunktu 1
        \item Rozwiązanie Autora 1 podpunktu 2
    \end{enumerate}
\end{mdframed}
\begin{mdframed}
    \begin{enumerate}
        \item Rozwiązanie Autora 2 podpunktu 1
        \item Rozwiązanie Autora 2 podpunktu 2
    \end{enumerate}
\end{mdframed}




\begin{zad}[Autor 1, Autor 2]
    Wykaż, że liczba drzew etykietowanych na zbiorze ${1, ..., n}$ wynosi $n^{n-2}$.
\end{zad}
\begin{mdframed}
    Rozwiązanie Autora 1.
\end{mdframed}
\begin{mdframed}
    Rozwiązanie Autora 2.
\end{mdframed}























\newpage
\section{Zestaw}          % ZESTAW 4

\begin{zad}[Autor 1, Autor 2]
    Oblicz $S(n, 2)$.
\end{zad}
\begin{mdframed}
    Rozwiązanie Autora 1.
\end{mdframed}
\begin{mdframed}
    Rozwiązanie Autora 2.
\end{mdframed}




\begin{zad}[Autor 1, Autor 2]
    Wykaż, że mamy dokładnie
    \[\frac{n!}{1^{\lambda_1} \cdot 2^{\lambda_2} \cdot  ... \cdot n^{\lambda_n} \cdot \lambda_1! \cdot ... \cdot \lambda_n!}\]
    permutacji zbioru $[n]$ o typie $1^{\lambda_1} \cdot 2^{\lambda_2} \cdot ... \cdot n^{\lambda_n} $  (mających $\lambda_i$ cykli długości i dla $i \in [n]$).
\end{zad}
\begin{mdframed}
    Rozwiązanie Autora 1.
\end{mdframed}
\begin{mdframed}
    Rozwiązanie Autora 2.
\end{mdframed}




\begin{zad}[Autor 1, Autor 2]
    Posługując się interpretacją kombinatoryczną, wykaż tożsamość:
    \[S(n+1,m+1) = \sum_k \binom{n}{k}S(k,m)\]
\end{zad}
\begin{mdframed}
    Rozwiązanie Autora 1.
\end{mdframed}
\begin{mdframed}
    Rozwiązanie Autora 2.
\end{mdframed}




\begin{zad}[Autor 1, Autor 2]
    Zakładając, że zachodzi równość:
    \[
        (x_1 + ... + x_k)^n = \sum_{n_1+...+n_k=n}\binom{n}{n_1 n_2 ... n_k}x_1^{n_1}\cdot...\cdot x_k^{n_k}
    \]
    podaj ile wynosi $\binom{n}{n_1 n_2 ... n_k}$.
\end{zad}
\begin{mdframed}
    Rozwiązanie Autora 1.
\end{mdframed}
\begin{mdframed}
    Rozwiązanie Autora 2.
\end{mdframed}




\begin{zad}[Autor 1, Autor 2]
    Wykaż, że
    \[
        \sum_{i=0}^{n} i \left[ n \atop i \right] = n! H_n,
    \]
    gdzie $H_n = 1 + \frac{1}{2} + ... + \frac{1}{n}$.
\end{zad}
\begin{mdframed}
    Rozwiązanie Autora 1.
\end{mdframed}
\begin{mdframed}
    Rozwiązanie Autora 2.
\end{mdframed}




\begin{zad}[Autor 1, Autor 2]
    Wykaż, że dla dowolnego $x \in \mathbb{R}$ zachodzi:
    \begin{enumerate}
        \item $x^n = \sum_{k}S(n,k) x^{\underline{k}}$
        \item $x^{\bar{n}} = \sum_{k} \left[n \atop k \right] x^k$.
    \end{enumerate}
\end{zad}
\begin{mdframed}
    \begin{enumerate}
        \item Rozwiązanie Autora 1 podpunktu 1
        \item Rozwiązanie Autora 1 podpunktu 2
    \end{enumerate}
\end{mdframed}
\begin{mdframed}
    \begin{enumerate}
        \item Rozwiązanie Autora 2 podpunktu 1
        \item Rozwiązanie Autora 2 podpunktu 2
    \end{enumerate}
\end{mdframed}















\newpage
\section{Zestaw}          % ZESTAW 5

\begin{zad}[Autor 1, Autor 2]
    Wykaż zasadę włączeń i wyłączeń korzystając z indukcji po liczbie zbiorów.
\end{zad}
\begin{mdframed}
    Rozwiązanie Autora 1.
\end{mdframed}
\begin{mdframed}
    Rozwiązanie Autora 2.
\end{mdframed}




\begin{zad}[Autor 1, Autor 2]
    Wykaż, że mamy
    \[
        \sum_{j=0}^{m}(-1)^j \binom{m}{j}(m-j)^n
    \]
    suriekcji ze zbioru $[n]$ w zbiór $[m]$.
\end{zad}
\begin{mdframed}
    Rozwiązanie Autora 1.
\end{mdframed}
\begin{mdframed}
    Rozwiązanie Autora 2.
\end{mdframed}



\begin{zad}[Autor 1, Autor 2]
    Ile jest ciągów długości $2n$ takich, że każda liczba $i \in [n]$
    występuje dokładnie dwa razy oraz każde sąsiednie dwa wyrazy są różne.
\end{zad}
\begin{mdframed}
    Rozwiązanie Autora 1.
\end{mdframed}
\begin{mdframed}
    Rozwiązanie Autora 2.
\end{mdframed}




\begin{zad}[Autor 1, Autor 2]
    Wykaż, że dla $n \geq 3$ zachodzi tożsamość
    \[
        D(n) = (n-1)(D(n-1) + D(n-2))
    \]
    gdzie $D(n)$ jest liczbą permutacji zboru $[n]$ bez punktów stałych.
\end{zad}
\begin{mdframed}
    Rozwiązanie Autora 1.
\end{mdframed}
\begin{mdframed}
    Rozwiązanie Autora 2.
\end{mdframed}




\begin{zad}[Autor 1, Autor 2]
    Wykaż (najlepiej kombinatorycznie), że dla dowolnych $n, k \in \mathbb{N}$
    zachodzi:
    \begin{enumerate}
        \item $S(n, k) = \sum_{0 \leq m_1 \leq m_2 \leq ... \leq m_{n-k} \leq k} m_1m_2 \cdot ... \cdot m_{n-k}$
        \item $c(n, k) = \sum_{0 < m_1 < m_2 < ... < m_{n-k} < k} m_1m_2 \cdot ... \cdot m_{n-k}$
    \end{enumerate}
\end{zad}
\begin{mdframed}
    \begin{enumerate}
        \item Rozwiązanie Autora 1 podpunktu 1
        \item Rozwiązanie Autora 1 podpunktu 2
    \end{enumerate}
\end{mdframed}
\begin{mdframed}
    \begin{enumerate}
        \item Rozwiązanie Autora 2 podpunktu 1
        \item Rozwiązanie Autora 2 podpunktu 2
    \end{enumerate}
\end{mdframed}


\begin{zad}[Autor 1, Autor 2]
    Ciąg podziałów zbioru ${1, ..., n}$ tworzymy następująco. Startujemy
    od podziału zawierającego tylko zbiór ${1, ..., n}$. Podział $(i + 1)$--wszy
    otrzymujemy z podziału $i$-tego poprzez:
    \begin{enumerate}
        \item wybranie jednego, co najmniej $2$-elementowego zbioru z podziału $i$-tego i podzielenie
              go na dwa niepuste podzbiory,
        \item podzielenie każdego, co najmniej $2$-elementowego zbioru z podziału $i$-tego na dwa
              niepuste podzbiory.
    \end{enumerate}
    W obu przypadkach procedura kończy swoje działanie jeżeli wszystkie zbiory podziału są
    jednoelementowe. Na ile sposobów można wykonać powyższe procedury? Na ile sposobów
    możemy wykonać powyższe procedury zakładając, że po każdym kroku zbiory podziałów
    zawierają kolejne liczby naturalne?
\end{zad}
\begin{mdframed}
    Rozwiązanie Autora 1.
\end{mdframed}
\begin{mdframed}
    Rozwiązanie Autora 2.
\end{mdframed}


\newpage
\section{Zestaw}          % Zestaw 6
Nothing here. Tylko work in progress...
\section{Zestaw}          % Zestaw 7
Nothing here. Tylko work in progress...
\section{Zestaw}          % Zestaw 8
Nothing here. Tylko work in progress...
\section{Zestaw}          % Zestaw 9
Nothing here. Tylko work in progress...
\section{Zestaw}          % Zestaw 10
Nothing here. Tylko work in progress...
\section{Zestaw}          % Zestaw 11
Nothing here. Tylko work in progress...
\section{Zestaw}          % Zestaw 12
Nothing here. Tylko work in progress...
\section{Zestaw}          % Zestaw 13
Nothing here. Tylko work in progress...






\end{document}
