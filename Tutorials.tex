\documentclass[12pt,a4paper]{article}
\usepackage[utf8]{inputenc}
\usepackage[T1]{fontenc}
\usepackage{amsmath,amssymb,amsfonts}
\usepackage{enumitem}
\usepackage{fancyhdr}
\usepackage{geometry}

\geometry{a4paper, margin=1in}

\pagestyle{fancy}
\fancyhf{}
\fancyhead[L]{Discrete Mathematics}
\fancyhead[R]{Solutions}
\fancyfoot[C]{\thepage}

\title{Discrete Mathematics - Solutions}
\author{Student Name}
\date{\today}

\begin{document}

\maketitle

\section{Set 1: Fundamentals}

\subsection*{Exercise 1.1}
\textit{(Regions formed by n lines on a plane)}

\vspace{1cm}
% Space for solution

\subsection*{Exercise 1.2}
\textit{(Fibonacci sequence properties)}

\begin{enumerate}[label=(\roman*)]
\item \textit{Sum of first n+1 Fibonacci numbers}
\vspace{1cm}
% Space for solution

\item \textit{Divisibility property}
\vspace{1cm}
% Space for solution

\item \textit{Upper bound}
\vspace{1cm}
% Space for solution
\end{enumerate}

\subsection*{Exercise 1.3}
\textit{(Tournament graph property - two-step reachability)}

\vspace{1cm}
% Space for solution

\subsection*{Exercise 1.4}
\textit{(Hamiltonian path in tournaments)}

\vspace{1cm}
% Space for solution

\subsection*{Exercise 1.5}
\textit{(Virus spread on chessboard)}

\vspace{1cm}
% Space for solution

\subsection*{Exercise 1.6}
\textit{(Equal number of acquaintances)}

\vspace{1cm}
% Space for solution

\subsection*{Exercise 1.7}
\textit{(Ambassadors and flags problem)}

\vspace{1cm}
% Space for solution

\subsection*{Exercise 1.8}
\textit{(Subsequence with sum divisible by n)}

\vspace{1cm}
% Space for solution

\subsection*{Exercise 1.9}
\textit{(Subset containment in large families)}

\vspace{1cm}
% Space for solution

\subsection*{Exercise 1.10}
\textit{(Element in at least half of large subsets)}

\vspace{1cm}
% Space for solution

\subsection*{Exercise 1.11}
\textit{(Covering chessboard with 2×2 and 3×3 tiles)}

\vspace{1cm}
% Space for solution

\subsection*{Exercise 1.12}
\textit{(Covering 2ⁿ×2ⁿ chessboard with L-shaped tiles)}

\vspace{1cm}
% Space for solution

\section{Set 2: Binomial Coefficients and Counting}

\subsection*{Exercise 2.1}
\textit{(Non-attacking rooks on n×n chessboard)}

\vspace{1cm}
% Space for solution

\subsection*{Exercise 2.2}
\textit{(Non-attacking rooks on n×m chessboard)}

\vspace{1cm}
% Space for solution

\subsection*{Exercise 2.3}
\textit{(Recurrence relations)}

\begin{enumerate}[label=(\roman*)]
\item \textit{Words without consecutive 1s}
\vspace{1cm}
% Space for solution

\item \textit{Domino tilings of 2×n rectangle}
\vspace{1cm}
% Space for solution
\end{enumerate}

\subsection*{Exercise 2.4}
\textit{(Solutions to equation x₁+x₂+x₃+x₄=7)}

\begin{enumerate}[label=(\roman*)]
\item \textit{Non-negative integer solutions}
\vspace{1cm}
% Space for solution

\item \textit{Positive integer solutions}
\vspace{1cm}
% Space for solution
\end{enumerate}

\subsection*{Exercise 2.5}
\textit{(Ways to cut out k×k squares from m×n chocolate)}

\vspace{1cm}
% Space for solution

\subsection*{Exercise 2.6}
\textit{(Summation rule for upper index)}

\vspace{1cm}
% Space for solution

\subsection*{Exercise 2.7}
\textit{(Parallel summation rule)}

\vspace{1cm}
% Space for solution

\subsection*{Exercise 2.8}
\textit{(Monotonic functions count)}

\vspace{1cm}
% Space for solution

\subsection*{Exercise 2.9}
\textit{(k-element subsets without adjacent numbers)}

\vspace{1cm}
% Space for solution

\subsection*{Exercise 2.10}
\textit{(Combinatorial identity proof)}

\vspace{1cm}
% Space for solution

\subsection*{Exercise 2.11}
\textit{(Multiple combinatorial identities)}

\begin{enumerate}[label=(\alph*)]
\item First identity
\vspace{1cm}
% Space for solution

\item Second identity
\vspace{1cm}
% Space for solution

\item Third identity
\vspace{1cm}
% Space for solution
\end{enumerate}

\section{Set 3: Stirling Numbers and Catalan Numbers}

\subsection*{Exercise 3.1}
\textit{(Unimodality of Stirling numbers of the second kind)}

\vspace{1cm}
% Space for solution

\subsection*{Exercise 3.2}
\textit{(Bell number recurrence relation)}

\vspace{1cm}
% Space for solution

\subsection*{Exercise 3.3}
\textit{(Stirling numbers identity)}

\vspace{1cm}
% Space for solution

\subsection*{Exercise 3.4}
\textit{(Numerical procedure and Bell numbers)}

\vspace{1cm}
% Space for solution

\subsection*{Exercise 3.5}
\textit{(Partitions without adjacent numbers)}

\vspace{1cm}
% Space for solution

\subsection*{Exercise 3.6}
\textit{(Binary trees and Catalan numbers)}

\vspace{1cm}
% Space for solution

\subsection*{Exercise 3.7}
\textit{(Triangulations of a convex polygon)}

\begin{enumerate}[label=(\roman*)]
\item Total triangulations
\vspace{1cm}
% Space for solution

\item Triangulations with boundary edges
\vspace{1cm}
% Space for solution
\end{enumerate}

\subsection*{Exercise 3.8}
\textit{(Labeled trees count)}

\vspace{1cm}
% Space for solution

\section{Set 4: Stirling Numbers and Bell Numbers}

\subsection*{Exercise 4.1}
\textit{(Calculate S(n,2))}

\vspace{1cm}
% Space for solution

\subsection*{Exercise 4.2}
\textit{(Permutations by cycle structure)}

\vspace{1cm}
% Space for solution

\subsection*{Exercise 4.3}
\textit{(Combinatorial identity for Stirling numbers)}

\vspace{1cm}
% Space for solution

\subsection*{Exercise 4.4}
\textit{(Multinomial coefficient value)}

\vspace{1cm}
% Space for solution

\subsection*{Exercise 4.5}
\textit{(Harmonic number sum identity)}

\vspace{1cm}
% Space for solution

\subsection*{Exercise 4.6}
\textit{(Falling and rising factorial identities)}

\vspace{1cm}
% Space for solution

\section{Set 5: Inclusion-Exclusion Principle and Counting}

\subsection*{Exercise 5.1}
\textit{(Prove inclusion-exclusion principle by induction)}

\vspace{1cm}
% Space for solution

\subsection*{Exercise 5.2}
\textit{(Count of surjections)}

\vspace{1cm}
% Space for solution

\subsection*{Exercise 5.3}
\textit{(Sequences with each number appearing twice without adjacency)}

\vspace{1cm}
% Space for solution

\subsection*{Exercise 5.4}
\textit{(Derangement recurrence relation)}

\vspace{1cm}
% Space for solution

\subsection*{Exercise 5.5}
\textit{(Combinatorial representations)}

\vspace{1cm}
% Space for solution

\subsection*{Exercise 5.6}
\textit{(Partition procedures)}

\vspace{1cm}
% Space for solution

\end{document}
